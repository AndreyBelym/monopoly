\input template.tex
\initESKD{Сетевое программирование}
\begin{document}
\newcounter{N}
\selectlanguage{russian}
\setcounter{page}{2}
\normalfont
\tableofcontents
\clearpage
\section*{ВВЕДЕНИЕ}
\addcontentsline{toc}{section}{ВВЕДЕНИЕ}
Компьютерные сети представляют собой одно из самых мощных и универсальных средств коммуникации.

Они позволяют создавать программы, связывающие пользователей, находящихся в различных точках земного шара для различных целей: обмена информацией, решения научных и производственных задач, развлечений.

Одной из самой интенсивной областей развития сетевого программирования является программирование сетевых компьютерных игр.

В данной работе обсуждается создание сетевой компьютерной версии классической игры "Монополия", и приводится программа на С++, реализующая клиентскую и серверную части игры. Отчёт содержит также описание функций, инструкцию пользователю и тестовый пример.
\clearpage
\section{СЕТЕВАЯ ИГРА МОНОПОЛИЯ}
\subsection{Содержательное описание задачи}
Создать приложение, позволяющие провести партию игры "Монополия" между игроками за компьютерами, связанными в сеть TCP/IP.
\subsection{Формальная постановка задачи}
Необходимо создать клиент и сервер для компьютерной версии игры "Монополия", позволяющую игрокам удобно создать и провести сеанс игры.

{\bf Краткое описание игры.}

Монополия — это классическая игра, в которой вы можете покупать, арендовать и продавать свою собственность! В начале игры участники выставляют свои фишки на поле «Вперед», затем перемещают их по игровому полю в зависимости от выпавшего на кубиках количества очков.

Если вы попадаете на Участок Недвижимости, которая пока еще никому не принадлежит, то вы можете купить эту Недвижимость у Банка. Если вы решаете не покупать ее, она может быть продана на Аукционе другому игроку, предложившему за нее самую высокую цену. Игроки, имеющие Недвижимость, могут взимать арендную плату с игроков, которые попадают на их Участок. При строительстве Домов и Гостиниц арендная плата значительно возрастает, поэтому вам следует вести строительсто на как можно большем количестве Участков.

Если вы нуждаетесь в деньгах, вы можете заложить вашу Недвижимость.

В ходе игры вам следует всегда выполнять указания, написанные на карточках «Общественная казна» и «Шанс». Но не расслабляйтесь — в некоторых случаях вас могут посадить в Тюрьму.

{\bf Цель}

Остаться единственным необанкротившимся игроком.

{\bf Начало игры}

Фишки всех игроков выстраиваются на поле «Вперед», после чего поочередно каждый делает свой ход.

{\bf Ход игры}

Когда подошла ваша очередь, бросте кубики. Ваша фишка передвинется по доске вперед в направлении по часовой стрелке. Поле, на котором вы остановитесь, определяет, что вам надо делать. На одном поле одновременно могут находиться несколько фишек. В зависимости от того на каком поле вы оказались, вам предстоит:

1) купить участки для строительства или другую недвижимость

2) заплатить арендную плату, если вы оказались на территории недвижимости, принадлежащей другим игрокам

3) уплатить налоги

4) вытащить карточку «Шансов» или «Общественной Казны»

5) оказаться в тюрьме

6) отдохнуть на «Бесплатной стоянке»

7) получить зарплату в размере 200\$


{\bf Прохождение поля «Вперед»}

Всякий раз, когда вы останавливаетесь или проходите поле «Вперед», двигаясь по часовой стрелке, Банк выплачивает вам зарплату 200 000. Эту сумму можно получить дважды за один и тот же ход, если, например, вы оказались на поле «Шанс» или «Общественная казна» сразу после поля «Вперед» и вытащили карточку с надписью «Перейдите на поле вперед».

{\bf Покупка недвижимости}

Если вы остановились на поле, обозначающем незанятую другими Недвижимость (то есть на участке для строительства не занятом ни одним из игроков), у вас будет право первого покупателя на его покупку. Если вы решили купить недвижимость, заплатите Банку деньги в сумме, указанной на игровом поле. В обмен вы получите право собственности на эту недвижимость (игровое поле окрасится в цвет вашей фишки).

Если вы решили не покупать Недвижимость, она немедленно выставляется на аукцион. В этом случае ее приобретает тот из игроков, кто предложит за нее наибольшую цену. Отказавшийся от покупки Недвижимости игрок не принимает участия в торгах.

Если в результате аукциона ни один из игроков не купил (или не смог купить) Недвижимость, то она остается свободной.

{\bf Владение недвижимостью}

Владение недвижимостью дает вам право взимать арендную плату с любых арендаторов, которые остановились на поле, обозначающем ее. Очень выгодно владеть недвижимостью всей цветовой группы — иными словами, владеть монополией. Если вы владеете всей цветовой группой, вы можете строить дома на любом участке Недвижимости этого цвета.

{\bf Остановка на чужой недвижимости}

Если вы останавливаетесь на чужой Недвижимости, которая была преобретена ранее другим игроком, с вас могут потребовать арендную плату за эту остановку. Сумма арендной плату недвижимости может изменяться в зависимости от построенных на поле этой недвижимост и домов и отелей. Если вся Недвижимость одной цветовой группы принадлежит одному игроку, арендная плата, взимаемая с вас за остановку на любом участке этой группы удваевается при условии, что на участках группы нет построек. Однако, если у владельца всей цветовой группы хотябы один участок Недвижимости этой группы заложен, он не может взимать с вас двойную арендную плату. Если на участках Недвижимости были построены Дома и Отели, арендная плата с этих участков увеличивается. За остановку на заложенной Недвижимости арендная плата не взимается.

{\bf Остановка на поле коммунального предприятия}

Если вы остановились на одном из таких полей, вы можете купить это Коммунальное предприятие, если оно еще никем не куплено. Как и при покупке другой недвижимости в этом случае вам придется уплатить Банку сумму, указанную на этом поле.

Если вы решили не покупать эту Недвижимость, Коммунальное предприятие выставляется на аукцион и продается игроку, предложившему за него наибольшую сумму. Вы не можете принять участие в аукционе.

Если в результате аукциона ни один из игроков не купил (или не смог купить) Коммунальное предприятие, то оно остается свободной.

Если это Коммунальное предприятие уже приобретено другим игроком, он может потребовать с Вас арендную плату. Арендная плата такого предприятия составит четырехкратное количество очков, выпавших на кубиках (вы снова кидаете кубики, чтобы определить сумму арендной платы). Если игрок владеет обоими Коммунальными предприятиями, вы должны будете заплатить ему сумму, равную десятикратному колличеству выпавших очков.

{\bf Остановка на вокзале}

Если вы первым остановились на таком поле, у вас будет возможность купить этот вокзал. При вашем нежелании приобретать Вокзал, он уходит на аукцион и продается игроку, предложившему за него наибольшую сумму. Вы не можете принять участие в аукционе.

Если в результате аукциона ни один из игроков не купил (или не смог купить) Вокзал, то он остается свободным.

Если у Вокзала уже есть хозяин, оказавшийся на нем должен заплатить арендную плату. Эта плата зависит от количества вокзалов у игрока-владельца вокзала, на котором вы остановились. Чем больше вокзалов у хозяина, тем плата больше.

{\bf Остановка на поле «Шанс» и «Общественная казна»}

Остановка на таком поле означает, что вам достается одна из карточек соответствующей группы. Эти карточки могут потребовать, чтобы вы:

1) передвинули вашу фишку

2) заплатили деньги, например, налоги

3) получили деньги

4) отправились в Тюрьму

Вы должны немедленно выполнить указания, написанные на карточке. Если вы взяли карточку, на которой написано «бесплатно освободитесь из тюрьмы», вы можете оставить ее у себя до тех пор, пока она вам не понадобится, или же вы можете продать ее другому игроку по договорной цене.

Примечание: На карточке может быть написано, что вы должны переместить фишку на другое поле. Если в процессе движения вы пересекаете по часовой стрелке поле «Вперед», то получите 200\$. Если вас отправляют в Тюрьму, то поле «Вперед» вы не пересекаете.

{\bf Остановка на поле налогов}

Если вы остановились на таком поле, вам просто нужно уплатить соответствующую сумму в банк.

{\bf Бесплатная стоянка}

Если вы остановились на таком поле, то просто отдохните до следующего вашего хода. Вы находитесь здесь бесплатно и не подвергаетесь никаким штрафам.

{\bf Тюрьма}

Вас отправляют в Тюрьму, если:

1) Вы остановились на поле «Отправляйтесь в Тюрьму», или

2) Вы взяли карточку «Шанса» или «Общественной Казны», на которой написано «Отправляйтесь в Тюрьму», или

Если вы попадаете в Тюрьму по карточке, зарплата в размере 200\$ вам не выплачивается, где бы вы до того не находились.

Чтобы выйти из Тюрьмы вам надо заплатить штраф в размере 50\$ и продолжить игру.

После того, как вы пропустили три хода, находясь в Тюрьме, вы должны выйти из нее и уплатить 50\$, прежде чем вы сможете передвинуть вашу фишкуна выпавшее на кубиках число полей.

Находясь в Тюрьме вы имеете право взимать арендную плату за вашу Недвижимость, если она не заложена. Если вы не были отправлены в Тюрьму, а просто остановились на поле Тюрьма в ходе игры, вы не платите штраф, так как вы «просто посетили» ее. Следующим ходом вы можете двигаться дальше, как обычно.

{\bf Дома}

После того, как вы собрали все участки Недвижимости одной цветовой группы, вы можете покупать Дома, чтобы поставить их на любом из имеющихся у вас участков. Это увеличивает арендную плату, которую вы можете взимать с арендаторов, останавливающихся на вашей Недвижимости. Вы можете покупать дома во время вашего хода перед броском кубика. Стоимость дома варьируется в зависимости от линии, к которой принадлежат цветовые группы Недвижимости. За один ход вы можете построить не более одного дома на полях принадлежащих одной цветовой группе.

Максимальное количество домов на одном участке — четыре.

Так же при необходимости вы можете продавать дома обратно в банк. Стоимость дома в этом случае будет такой же, за какую вы его приобретали.

Нельзя строить дома, если хотя бы один участок данной цветовой группы заложен.

{\bf Отели}

Прежде, чем вы сможете покупать отели, вам нужно иметь четыре дома на участке, на котором вы собираетесь построить отель. Отели покупаются также, как и дома, по той же цене. При воздвижении отеля, четыре дома с этого участка возвращаются в банк. На каждом участке можно построить только один отель.

{\bf Продажа недвижимости}

Вы можете продать незастроенные участки, железнодорожные вокзалы и коммунальные предприятия любому игроку, заключив с ним частную сделку, на сумму согласованную между вами. Если на продоваемых вами участках имеются дома или отели, то продавать такую недвижимость нельзя. Сперва необходимо продать банку дома и отели стоящие на всех участках этой цветовой группы, а только после этого предлагать сделку другому игроку.

В сделке с обеих сторон для обмена могут быть предложены как участки Недвижимости, так и деньги, и карточки освобождения из тюрьмы. Комбинации обмена могут быть самыми разнообразными на усмотрение игроков. Если игроку не интересна предложенная сделка, он может отказаться от нее.

Ни дома, ни отели нельзя продавать другим игрокам. Их можно продавать только банку. 

При необходимости, для того, чтобы вы могли получить деньги, отели могут быть снова заменены домами. Для этого вам нужно продать отель в банк и получить взамен четыре дома, плюс стоимость самого отеля.

{\bf Залог}

Если у вас не осталось денег, но могут возникнуть долги, вы можете получить деньги, заложив какую-либо Недвижимость или продать дома или отели. Для того, чтобы заложить недвижимость, необходимо сначала продать все дома и отели, построенные на участках закладываемой цветовой группы. При залоге, вы получаете из банка сумму, равную половине стоимости закладываемого участка. Если позднее вы захотите выкупить заложенную Недвижимость, вам придется выплатить банку ее полную стоимость, плюс 10% сверху.

Если вы закладываете какую-либо Недвижимость, она попрежнему принадлежит вам. Ни один игрок не вправе выкупить ее вместо вас у банка.

С заложеной Недвижимости нельзя взимать арендную плату, хотя арендная плата попрежнему может поступать к вам за другие объекты Недвижимости той же цветовой группы.

Вы не можете продавать заложенную Недвижимость другим игрокам.

Возможность строить на участках дома появляется только после выкупа всех без исключения участков одной цветовой группы.

{\bf Банкротство}

Если вы должны банку или другим игрокам больше денег, чем вы можете получить по вашим игровым активам, вас объявляют банкротом, и вы выбываете из игры.

Если вы должны банку, банк получает все ваши деньги и всю вашу Недвижимость. Вернувшаяся в банк недвижимость поступает в свободную продажу. Также в банк возвращаются карточки освобождения от тюрьмы.

Если вы обанкротились из-за долгов другому игроку, все ваше имущество отправляется в банк. Вернувшаяся в банк недвижимость поступает в свободную продажу. А вашему должнику Банк выплачивает сумму долга.

Так же вы можете стать банкротом, если не успеете выполнить какое-либо игровое действие в отведенное на него время.

{\bf Победитель}

Последний оставшийся в игре участник является победителем.
\section{РАЗРАБОТКА АЛГОРИТМА}
\subsection{Разработка графического интерфейса пользователя}
Для обеспечения минимального порога вхождения основной интерфейс должен более-менее точно копировать реальную доску игры "Монополия".

Основной экран обрамлен 40 стандартными полями игры - улицы, налоги, вокзалы и т.д. Рядом с каждым полем-улицей, ближе к центру экрана, находится прямоугольник цветовой группы, на котором будут располагаться дома и отели при покупке их игроком. Фишки, обозначающие положение игроков, находятся в нижних частях игровых полей. Если пользователь щелкает по полю, ему показывается информация по этому полю - рента для улиц, стоимость налога и т.д.

В оставшейся части экрана отображается список игроков, укрупненно показывается его фишка, имя, тип или состояние - локальный, удаленный, отключен и т.д., баланс. Кроме того, здесь располагаются 3 кнопки - завершения хода, отключения и выхода.

Помимо окна с информацией о полях, следует использовать окна для подключения, для присоединения игрока к игре, аукциона и торговли и различных сообщений.

Итак, внешний вид разработанного интерфейса представлен на рисунке \ref{INT1}.
ы\pic{INT1.png}{Разработанный интерфейс программы}{INT1}{H}
\subsection{Разработка структур данных}
Сервер и Клиент обмениваются сообщениями, структура которых имеет следующий формат:
        
        type - тип сообщения(расшифровывается ниже);
        
        names - массив имен игроков;
        
        src,dst - обычно начальное и конечное поле события;
        
        dice1,dice2 - выпавшие кости;
        
        curr\_player - текущий игрок;
        
        pnum - номер пакета;
        
        moneys - балансы игроков;

{\bf Типы сообщений:}

EVENT - игровое событие - переход игрока на новое поле,

DISCONNECT - отключение сервера или клиента,

ACK - подтверждение  соединения между клиентом и сервером,

SERVER\_FULL - указывает, что сервер переполнен,

UPDATE\_PLAYERS - указание клиентам обновить информацию об игроках,

UPDATE\_FIELD - указание обновить информацию об игровом поле,

OWNED - говорит, что игрок покупает текущее поле,

NOT\_OWNED - говорит, что игрок не покупает текущее поле,,

LOOSER - указание сервера на проигравшего игрока,

WINNER - указание сервера на выигравшего игрока,

AUCTION - сообщение об установке цены на аукционах и торгах,

BUY\_HOUSE - запрос на покупку отеля,

BUY\_HOTEL - запрос на продажу отеля,

BUY\_FIELD - запрос на покупку поля,

SELL\_FIELD - запрос на продажу поля.
\subsection{Разработка структуры алгоритма}
Программу можно разбить на следующие части:

1) Сервер - выполняет функции Банкира и Аукционера, а также кидает кости. Сам сервер может быть разбит на части: подпрограмма service\_thread,которая управляет подключением игроков к серверу; подпрограмма main\_server\_thread, которая имитирует кидание костей и подсчет баланса игроков; подпрограмма server\_thread, которая получает и обрабатывает сообщения конкретного игрока; подпрограммы auct\_thread и trade\_thread, которые позволяет проводить аукционы и торговлю между игроками.

2) Клиент client\_thread - получает сообщения от сервера, обновляет игровые данные и запрашивает отрисовку.

3) Части приложения, которые отрисовывают данные, получаемые клиентом.
\section{РАЗРАБОТКА ПРОГРАММЫ}
\subsection{Описание переменных и структур данных}
Сервер и Клиент обмениваются сообщениями типа PacketData, структура которых имеет следующий формат:
        
       PacketType type - тип сообщения(расшифровывается ниже);
        
       char names[168] - массив имен игроков;
        
       Sint32 src,dst - обычно начальное и конечное поле события;
        
       Sint32 dice1,dice2 - выпавшие кости;
        
       Sint32 curr\_player - текущий игрок;
        
       Uint32 pnum - номер пакета;
        
       Uint32 moneys[8] - балансы игроков;

{\bf Типы сообщений:}

EVENT - игровое событие - переход игрока на новое поле,

DISCONNECT - отключение сервера или клиента,

ACK - подтверждение  соединения между клиентом и сервером,

SERVER\_FULL - указывает, что сервер переполнен,

UPDATE\_PLAYERS - указание клиентам обновить информацию об игроках,

UPDATE\_FIELD - указание обновить информацию об игровом поле,

OWNED - говорит, что игрок покупает текущее поле,

NOT\_OWNED - говорит, что игрок не покупает текущее поле,,

LOOSER - указание сервера на проигравшего игрока,

WINNER - указание сервера на выигравшего игрока,

AUCTION - сообщение об установке цены на аукционах и торгах,

BUY\_HOUSE - запрос на покупку отеля,

BUY\_HOTEL - запрос на продажу отеля,

BUY\_FIELD - запрос на покупку поля,

SELL\_FIELD - запрос на продажу поля.
\subsection{Описание функций}
\elist{
\item main\_server\_thread(void *data)

Подпрограмма, которая имитирует кидание костей и подсчет баланса игроков;

\item server\_thread(void *data)

Подпрограмма, которая получает и обрабатывает сообщения конкретного игрока; 

\item service\_thread(void *data)

Подпрограмма, которая управляет подключением игроков к серверу;

\item auct\_thread(void *data)

Подпрограмма, которая позволяет проводить аукционы

\item trade\_thread(void *data)

Подпрограмма, которая организует торговлю между игроками.

\item client\_thread(void *data)

Подпрограмма, которая получает сообщения от сервера, обновляет игровые данные и запрашивает отрисовку.

\item redraw()

Подпрограмма, которая перерисовывает экран приложения.

\item redraw\_board()

Подпрограмма, которая перерисовывает игровое поле.

}
\section{ИНСТРУКЦИЯ ПОЛЬЗОВАТЕЛЮ}
Данная программа позволяет сыграть партию игры "Монополия" с игроками за локальным компьютером или находящимися на удаленных компьютерах.

Для начала подключитесь к серверу, введя его символический или цифровой адрес и порт. Вы также можете создать сервер на указанном порту для указанного количества игроков.

После добавьте пользователей, сидящих за вашим компьютером к списку игроков сервера. Когда все места для игроков будут заняты, сервер начнет игру.

Вы можете руководствоваться стандартными правилами "Монополии". После того, как сервер передвигает вас на следующее поле, вы можете покупать улицы, воказалы, предприятия,дома и т.д. При проведении торгов и аукциона используется таймер на 9 секунд. После того, как первый игрок устанавливает цену, таймер активизируется. Когда таймер обнуляется, аукцион или торги заканчиваются. Каждое обновление цены игроками сбрасывает таймер обратно на 9 секунд.

Вы можете нажать Ctrl+F, чтобы переключаться в полноэкранный режим и обратно.

Для выхода закройте окно, нажмите соответствующую кнопку на экране или клавишу Esc на клавиатуре.  
\section{ТЕСТОВЫЙ ПРИМЕР}
Ниже на рисунке \ref{SCR1} представлен пример работы программы-игры "Монополия".
\pic{SCR1.png}{Пример работы программы-игры "Монополия"}{SCR1}{H}
\section*{ЗАКЛЮЧЕНИЕ}
\addcontentsline{toc}{section}{ЗАКЛЮЧЕНИЕ}
Программирование сетевых приложений - одна из самых динамично развивающихся областей программирования.

Создание сетевой компьютерной версии игры "Монополия" позволяет понять общие принципы сетевого программирования, а также принципы создания компьютерных игр и сложных графических приложений. 
\section*{СПИСОК ИСПОЛЬЗОВАННЫХ ИСТОЧНИКОВ}
\addcontentsline{toc}{section}{СПИСОК ИСПОЛЬЗОВАННЫХ ИСТОЧНИКОВ}
1) http://www.libsdl.org

2) http://guichan.sourceforge.net

3) http://ru.wikipedia.org

4) http://en.wikipedia.org
\section*{ПРИЛОЖЕНИЕ}
\addcontentsline{toc}{section}{ПРИЛОЖЕНИЕ}
Далее приводится общая часть сетевой программы-игры "Монополия".
\prog{C++}{Monopoly/core.hpp}
\prog{C++}{Monopoly/core.cpp}
Далее приводится серверная часть сетевой программы-игры "Монополия".
\prog{C++}{Monopoly/server.hpp}
\prog{C++}{Monopoly/server.cpp}

Далее приводится клиентская часть сетевой программы-игры "Монополия".
\prog{C++}{Monopoly/client.hpp}
\prog{C++}{Monopoly/client.cpp}
Далее приводится часть сетевой программы-игры "Монополия", которая занимается отрисовкой экрана.
\prog{C++}{Monopoly/sdl.hpp}
\prog{C++}{Monopoly/sdl.cpp}

Далее приводится часть сетевой программы-игры "Монополия", которая занимается организацией GUI.
\prog{C++}{Monopoly/widgets.hpp}
\prog{C++}{Monopoly/widgets.cpp}

Далее приводится основная часть сетевой программы-игры "Монополия".
\prog{C++}{Monopoly/main.cpp}

\end{document}
